\documentclass{ecnreport}

\stud{Master 1 CORO / Option Robotique}
\topic{Robot Operating System (2)}
\author{O. Kermorgant}

\begin{document}

\inserttitle{Robot Operating System}

\insertsubtitle{Lab 1: Packages, launch files,  parameters and topic remapping}

\section{Goals}

In this lab we will see the main tools to analyze nodes and topics. A ROS 2 package will be used to move Baxter in a simulation environment.

\subsection{Using the terminals}

In ROS, a lot of commands have to be run from the terminal. Each terminal should be configured either to run ROS 1 things, or ROS 2 things.
A simple shortcut allows changing this:
\begin{bashcodelarge}
ros1ws  # type this to configure the terminal for ROS 1 (default)
ros2ws  # type this to configure the terminal for ROS 2 (has to be done manually)
\end{bashcodelarge}

Each time a package is compiled, the corresponding command (ros1ws / ros2ws) should be run in order to refresh the package list.

\subsection{Running the simulation (already installed)}

The Baxter simulator behaves as the actual Baxter, meaning that it works on ROS 1. A ROS 1 / ROS 2 bridge is used to forward messages between the two protocols.
The simulation should be run from a ROS 1 terminal with:
\begin{bashcodelarge}
roslaunch baxter_simple_sim simulation.launch
\end{bashcodelarge}

It will display RViz with a few in/out topics that can be listed through:
\begin{bashcodelarge}
rostopic list (ROS 1)
ros2 topic list (ROS 2)
\end{bashcodelarge}

\subsection{Compiling the package}

The folder should be put in your ROS 2 workspace (\texttt{\texttilde/ros2/src}). Compilation is done by calling colcon from the root of the workspace:
\begin{bashcodelarge}
 ros2ws
 cd ~/ros2
 colbuild
\end{bashcodelarge}

\section{Running the control nodes}

A basic control GUI can be run with:

\begin{bashcodelarge}
 ros2 launch move_joint slider_launch.py
\end{bashcodelarge}
It runs a node that sends the slider value on a topic (here \texttt{setpoint}).\\

The Baxter robot has $2\times 7$ joints as shown in \Fig{baxter}, the names of which are listed in the table.
\begin{figure}[h]\centering
 \includegraphics[width=.3\linewidth]{baxter} \\
  \begin{tabular}{|c|c|c|c|c|c|c|c|}
  \hline
  joints (\texttt{left\_} \& \texttt{right\_})& \texttt{s0} & \texttt{s1}& \texttt{e0} & \texttt{e1} & \texttt{w0} & \texttt{w1} & \texttt{w2} \\\hline
 \end{tabular}
 \caption{Baxter joints}
 \label{baxter}
\end{figure}

A special node called \texttt{move\_joint} should be used to change the value published by the slider GUI, to the setpoint of a particular joint:
\begin{bashcodelarge}
 ros2 run move_joint move_joint
\end{bashcodelarge}

\subsection{Initial state of the control nodes}

Use \texttt{ros2 node} and \texttt{ros2 topic} to get the information on the nodes. There are currently two limitations:
\begin{itemize}
 \item The \texttt{move\_joint} node needs a parameter to tell which joint it should be controlling
 \item The slider publishes on \texttt{setpoint} while the  \texttt{move\_joint} node listens to \texttt{joint\_setpoint}
\end{itemize}
Additionally, it would be nice to run the two nodes in the same lauch file.

\subsection{Regroup all nodes in the same launch file}

Open the \texttt{slider\_launch.py} and add a line equivalent to calling \texttt{ros2 run move\_joint move\_joint}.\\
Then, add a parameter to tell which joint should be controlled:\\\texttt{parameters = \{'joint\_name': 'right\_e0'\}}\\

To run this launch file, you can go to its folder and type \texttt{ros2 launch ./slider\_launch.py}.\\
Display the graph (\texttt{rqt\_graph}) to detect that the slider node and \texttt{move\_joint} do not communicate, because they do not use the same topics.

\subsection{Remapping}

In this section we will remap the topic name of \texttt{move\_joint} such that is uses \texttt{setpoint} instead of \texttt{joint\_setpoint}.\\

Modify the launch file by adding this argument to the \texttt{move\_joint} node:

\begin{pythoncodelarge}
remappings = {'joint_setpoint': 'setpoint'}.items()
\end{pythoncodelarge}
Run the launch file again and you should be able to control the chosen joint.

\section{Playing with launch files}

\subsection{Argument for joint name}

Now that a launch file exists to control 1 joint from a slider, we will change the hard-coded joint name to an argument:
\begin{pythoncodelarge}
sl.declare_arg('name', 'right_e0')
\end{pythoncodelarge}
This syntax tells the launch file that is now has a \texttt{name} argument, with default value \texttt{'right\_e0'}.\\

Change the hard-coded value to the argument one \texttt{sl.arg('right\_e0')} and check that the behavior is the same\\

Also, you can rename the slider GUI by giving an additional argument:
\begin{pythoncodelarge}
sl.node('slider_publisher', 'slider_publisher', name=sl.arg('name'), ...)
\end{pythoncodelarge}

Then, run the launch file with another name, for instance:\\ \texttt{ros2 launch ./slider\_launch.py name:=left\_e0}\\
On another terminal, try to run the same launch file for another joint. What happens?\\



\subsection{Including launch files in other launch files}

In this last section we will write a new launch file that will include the previous one, for various joint names. In order to avoid node / topic duplicates, each nodes relative to a specific joint will be in their own namespace. 

The syntax to include a launch file in another one is as follow:
\begin{pythoncodelarge}
sl.include('move_joint', 'slider_launch.py', launch_arguments = [('name', 'right_e0')])
\end{pythoncodelarge}

It should be called from a namespace block, with this syntax:
\begin{pythoncodelarge}
with sl.group(ns = 'right_e0'):
  sl.include('move_joint', 'slider_launch.py', launch_arguments = [('name', 'right_e0')])
\end{pythoncodelarge}

Check that the behavior is similar to the previous one. Then, write a for loop to run this code for many joint names:
\begin{pythoncodelarge}
joints = ('right_e0', 'right_e1', ...)
for joint in joints:
    with sl.group(ns = joint):
        sl.include(...)
\end{pythoncodelarge}



\end{document}
